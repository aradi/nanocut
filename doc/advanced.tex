\section{Advanced Usage}
\label{Advanced Usage}
\subsection{The order parameter - adding and substracting bodies}
In addition to simply creating structures of the shapes mentioned in the previous section, Nanocut features the possibility to substract and add those to build more complex shapes.

The \lstinline{order} parameter is an integer which can be specified for every body. It determines the order in which the bodies are substracted or added. The bodies are processed in ascending order, beginnig with the bodies with the smallest value for \lstinline{order} greater than zero. Therefore bodies with orders lower than one are ignored.
An uneven value for \lstinline{order} causes a body to be added. Those with even values for \lstinline{order} are substracted. The default value for \lstinline{order} is \lstinline{1}.

It is allowed and common practice to define multiple bodies with the same value for \lstinline{order}, since there is no difference in the result for processing a set of bodies in any order as long as they are all substracted or added.

\paragraph{Example}\ 
\lstinputlisting{srcexamples/order.ini}
\ \\\includegraphics[width=0.6\textwidth]{srcexamples/order.png}

\subsection{Shift vectors}
The \lstinline{shift_vector} parameter enables the possibility to shift a body to a certain position. The value for \lstinline{shift_vector} is the vector defining the translation.

This is particularly useful when the order parameter is used to create specific shapes.

\subsubsection{Periodic bodies}

Using a shift vector with periodic bodies can lead to unexpected results at first glance, for two reasons. 
\begin{enumerate}
 \item Every atom is moved into the first supercell after being cut out. This undoes the effect of the shift vector's components in the direction(s) of the axis or both axes.
 \item The automatic rotation of periodic bodies is applied after the body is shifted.
\end{enumerate}
In combination this means: 

A body's translation in 1D-periodic structures is visible in the result as a translation inside the x-y-plane which is lacking the component in direction of the periodicity axis.

In 2D-periodic structures the translation is visible in z-direction by the shift vector's component orthogonal to both axes.
